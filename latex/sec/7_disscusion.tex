\section{Discussion}
\label{sec:discussion}

Based on the R squared values and comparisons of the visual image outputs, the combined model performs poorly in comparison to KITRO on its own. The poor performance of the combined model may be explained by the claim that was made in the research paper where KITRO was introduced. The paper claims that KITRO's choice of using a single decision tree is justified by the fact that the decision tree traces all hypotheses and can be represented as a closed-form calculation, which allows it to always select the most probable option. Introducing bootstrapping and feature column randomization across multiple decision trees, then averaging the results, is what Random Forest regression does. This randomization and further modification of the original dataset turns KITRO from a closed-form calculation to an open-form calculation. For the combined model which uses 100 decision trees, each decision tree, which is bound to select a different result for several reasons, contributes to the averaged result which may be completely different from the one provided by KITRO on its own.

The inclusion of the refined vertices in the set of independent (X) variables may have also been a contributing factor in the poor performance of the combined model. The number of feature columns that belong to refined vertices makes up for more than four fifths of the entire dataset; the inclusion of refined vertices may have added too much noise during the training and testing of the regression model, regardless of the conditions they were added in. This is supported by the fact that the R squared values are greater for the cases where the refined vertices are excluded than the cases where they are included, although this may be a coincidence.

The output images from the visualization script show that, while the shape of the human mesh is generally reasonable, Random Forest regression fails to accurately model the pose of a human mesh. The combined model leaves much to be desired when it comes to copying most movements from KITRO's output. Most movements are contracted, and many movements might not have even carried over from KITRO's output, with a few clear movements such as stretching both arms to create T-pose, or bending down, being the exception.

The choppiness of the combined model in comparison to KITRO on its own may be a result of the significantly inaccurate predictions in both the pose and the camera feature values. Strangely, the shape of the human mesh, which represents the time-invariant dimensions and proportions of the body, seems to be mostly stable, since it resembles a human model that doesn't become warped for nearly the entire duration, and there is some semblance of human motion between the different frames of the GIF output. The occasional unnatural stretching and warping of the human mesh, on the other hand, particularly when bending over, may be a result of the shape being incorrectly predicted for that portion of the data.

Lastly, KITRO's visualized mesh output on its own appears to have fluid motion. For some portions of the data, the motion is slow and creepy, and the camera vibrates a bit, but the motion is pretty fluid nonetheless. The lack of changes in facial expressions, as mentioned in another section, is probably caused by KITRO only using SMPL-X for data conversion, instead of for its intended purpose of extending SMPL to account for facial expressions and hand movement, making it a limitation of KITRO's implementation, rather than the dataset or something more fundamental.