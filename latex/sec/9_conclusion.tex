\section{Conclusion}
\label{sec:conclusion}

Through the changes introduced by our project, we present a brief glimpse of the inner workings of human mesh recovery. We restructured the output parameters of KITRO into a format more readily accessible for Python regression models, which improves the overall accessibility of the KITRO model and makes it easier to experiment with the process of human mesh recovery. We trained the Random Forest regression model on the output of KITRO, in which the results establish a more concrete basis for KITRO's effectiveness as a model that should run using one decision tree. Lastly, we created scripts to help visualize the 3D human mesh outputs of SMPL, which is used by KITRO. Visualizing 3D meshes serves to improve the intuitiveness of working with human mesh recovery, and facilitating this process makes the project, as well as working with human mesh recovery in general, easier to interact with and more accessible to everyone.