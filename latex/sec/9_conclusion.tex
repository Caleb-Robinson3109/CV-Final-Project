\section{Conclusion}
\label{sec:conclusion}

The challenge of generating an accurate 3D human mesh from a 2D image was initially tackled by gradient-based models. KITRO, which stands for kinematic-tree rotation, is a newer solution for HMR that runs on top of SMPL and accounts for 2D pose, bone length, and proximal joint depth. Contrary to the solutions that came before it, KITRO emerged as the main mesh refinement model which stood out for its use of only a single decision tree which does not rely on any gradient-based refinements. The result of using a single decision tree and taking advantage of its closed-form nature is a model which accurately selects hypotheses for bone directions, and this claim is further supported by its performance in relation to the combined Random Forest regression + KITRO model.

Through the changes introduced by our project, we present a brief glimpse of the inner workings of human mesh recovery. For people familiar with machine learning or data mining, we restructured the output parameters of KITRO into a format more readily accessible for Python regression models, which improves the overall accessibility of the KITRO model and makes it easier to experiment with the process of human mesh recovery. For people familiar with HMR, we trained the Random Forest regression model on the output of KITRO, in which the results establish a more concrete basis for KITRO's effectiveness as a model that should run using one decision tree. Lastly, for people who may find HMR to be interesting, we created scripts to help visualize the 3D human mesh outputs of SMPL, which is used by KITRO. Visualizing 3D meshes may improve the intuitiveness of working with human mesh recovery, and facilitating this process makes the project, as well as working with human mesh recovery in general, easier to interact with and more accessible to everyone. Our contributions may serve as encouragement for people who are interested in computer vision, virtual reality, real world simulations, or data processing as a real world application.