\begin{abstract}
The generation of a 3D human mesh from a 2D image is an important problem in the world of computer vision. There have been many solutions to provide the best model for this task, most of which rely on gradient-based algorithms. One model stands out as an exception: KITRO, which stands for kinematic-tree rotation, uses only decision trees, and does not make corrections via Gradient Descent. With KITRO's choice of using only decision trees, the question arises: is it possible to have a separate model stacked on top of KITRO to learn from the errors of its output, thereby minimizing KITRO's errors and further refining its predictions? In an attempt to refine the predictions of KITRO for 3D mesh parameter prediction, we record the model's output, and then we apply bootstrapping to that output to train a Random Forest regression model that has KITRO's refinements. This is then used to generate a 3D mesh with our custom visualization tool. We use Random Forest because it does not rely on Gradient Descent to learn. Many of the current models on predicting 3D meshes use Gradient Descent to correct their output as it's being produced, but this runs directly counter to KITRO's approach, which intentionally avoids correcting its previous responses. We found that the usage of Random Forest regression on the KITRO model hurts its results. Our results, however, support a claim that was made in the paper where KITRO \cite{KITRO_2024} was first introduced: KITRO, which uses closed-form calculations, is best implemented through decision trees, and gradient-based learning is hindered by the complexity of the data.
\end{abstract}