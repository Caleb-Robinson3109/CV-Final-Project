\begin{abstract}
Generation of a 3D human mesh from a 2D image is an important problem in the world of computer vision. There have been many solutions to provide the best model for this task. With the ability to train on predicted values of the model and the ground truth values, the question arises: is it possible to have a model stacked on top of the first model to learn from the errors of the first, thereby minimizing the errors and refining the original predictions? In order to refine the predictions of the KITRO model for 3D mesh parameter prediction, we record the output of the KITRO model, and apply bootstrapping to train a Random Forest regression model that has KITRO's refinements. This will then be used to generate a 3D mesh with our custom visualization tool. We use Random Forest because it does not utilize Gradient Descent to learn, which many of the models on predicting 3D meshes currently use. !!TODO add a sentence about results!! - We found that the model helps hurts does nothing which shows use this its not effective blah blah blah. These findings are significant because it furthers the understanding of 3D mesh generation models. Of which can be used in many applications.
\end{abstract}