\section{Related Work}
\label{sec:relatedword}

KITRO is the model that generates the predictions that the new model takes to perform further processing on, ultimately to be used in hopes of generating a more refined 3D mesh. KITRO uses 2D keypoints, as well as an HMR model that predicts twenty-four joints to refine the pose, bone length, and parent joint depth. The 2D keypoints are part of what makes the model excel, because it helps the model fine-tune the estimated 3D mesh. KITRO uses a singular decision tree to determine the most probable configuration of joints: at each branching path on the decision tree, the direction of the path is compared to a corresponding bone location on a human skeleton's kinematic-tree. The overall process is known as kinematic-tree rotation, hence the name of the model.

HMR, which stands for human mesh recovery, is the process of creating a 3D human mesh from a 2D image. It has applications in computer interactions with the real world, as well as virtual reality and simulations. In relation to KITRO, the HMR model produces an unrefined mesh, which is then processed by KITRO to refine the mesh.

SMPL, one of the dependencies used by KITRO, is a model for representing the human body by triangulating meshes, and is necessary for performing human mesh recovery. Similarly, SMPL-X is another dependency used by KITRO. While it is intended to extend the SMPL model to support hands and facial expressions within the model, KITRO does not appear to use SMPL-X for this purpose, and instead uses it for data conversion.