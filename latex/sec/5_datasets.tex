\section{Datasets}
\label{sec:datasets}

KITRO uses two datasets, and both are provided to us with some preprocessing already done to it. The first data set is 3DPW. The 3DPW dataset has many outdoor images tailored for 3D pose and shape estimation. The second dataset, Human3.6M, is an indoor 3D mesh dataset. The preprocessing that KITRO does is converting both into a common format. The converted format is a Python dictionary with the following values: image name, predicted 3D rotation matrix, predicted body shape parameters, predicted camera translations, intrinsic camera parameters, 2D keypoints, ground truth values of 3D rotation parameters, and ground truth values of body shape parameters.

We had to further transform the preprocessed data in order to properly work with it. The first step was to combine the two datasets into one overall dataset, and then take samples as large as one-hundredth of the new dataset. This was done because of hardware memory limits that prevented processing the entire dataset as one object. After that, we needed to run the KITRO model on the smaller samples. This generated the values for refined thetas, refined shape, refined camera, and refined vertices.

With one-hundred samples of the original given data, as well as the KITRO refinements applied to them, the next step was to combine the samples to create one new dataset. Due to a lack of computer memory, five separate datasets were created. To distribute the datasets among the five final datasets, every fifth dataset was combined. With five datasets, four were designated as training datasets and one as a testing dataset.

The final step was to flatten the dictionary. The name field was removed and all the data was flattened into one numpy array.