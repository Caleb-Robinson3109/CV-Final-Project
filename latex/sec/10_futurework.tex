\section{Future Work}
\label{sec:futurework}

Given that the project's aim is to explore human mesh recovery through improving KITRO's use cases, performance, and usability, there are numerous directions in which this project could be expanded. While working with the SMPL output, we came to the realization that SMPL-X is not used for its intended purpose of incorporating changes in facial expressions and hand movements into the SMPL model \cite{SMPL-X:2019}. To account for this, a future change could modify KITRO to use the additional facial expression and hand movement provided by SMPL-X, and potentially increase the number of joints from 59 joints to a greater quantity, if necessary.

Another possible direction that this project could head in is swapping out the Random Forest regression model with other regression models to test on KITRO's output, and evaluating their performance. It's possible that other models may perform significantly better than Random Forest regression, although it may be challenging to significantly refine the output of the KITRO model.

Instead of swapping the Random Forest regression model for other models, a future change could alter how the Random Forest regression model interacts with KITRO. Rather than running the Random Forest regressor implemented via scikit-learn, the KITRO model can be directly altered to perform bootstrapping on the original dataset, and run multiple KITRO decision trees with differing feature column selections across the bootstrapped sets of data. This runs through the same process as a Random Forest regression model, but through this method, the decision trees are entirely KITRO decision trees, and no manual configuration of feature columns is needed to tune the results. The results of each decision tree would then be averaged to select the average most likely outcome.

We may also achieve better results if we drastically increase the size of the training data. We only trained on subsets of the aggregate data due to time constraints, and settled with training on one-hundredth of the aggregate data by the end. With more time, it is possible that more extensive training with the current Random Forest implementation may yield results with a less unreasonable R squared similarity score.

It would be beneficial to migrate the KITRO model from PyTorch 1.11.0 to the newest version of PyTorch, since PyTorch introduces some performance optimizations in later versions. Additionally, migrating all dependencies to be compatible with the latest version of Python and using the latest version of important dependencies, such as Numpy, will serve to improve the accessibility of the project. The original recommended version of PyTorch for the KITRO project was 1.10.1, for which the version of Python that can run this is locked to Windows 10 or earlier. Swapping to PyTorch 1.11.0 allowed KITRO to be run on Windows 11; it is very possible that PyTorch will need to be updated if a new version of Windows (or another operating system) releases. At present, certain dependencies rely on outdated versions of other dependencies, which prevented us from updating PyTorch to the most recent version. One likely solution to this is to replace the outdated dependency with a more up-to-date alternative dependency. If an outdated dependency does not have any alternatives that can replace it, it may be necessary to create a forked version of the dependency in order to directly apply changes that will make it compatible with present standards. Another potential solution is to refactor the project's code to no longer rely on the dependency at all.

It may be useful to create a script that takes numerous images from the user as input to create a new dataset that KITRO can run on to generate refined 3D human meshes. The creation of new datasets for HMR may prove useful not only in training KITRO, but also in training other existing mesh refinement and mesh recovery models, and creating new models. It extends the breadth of the data from a limited view outdoor activity and indoor activity to anything the user is able to capture as an image, as long as it appears to be suitable for HMR.

Due to time constraints, as well as the ability to make visual comparisons, we neglected to consider that other metrics may have been more appropriate to compare the similarity between meshes than the R squared score. In the future, it may be more appropriate to take measurements of the MPJPE (Mean Per Joint Position Error), or PA-MPJPE, which is a finer version of MPJPE, and PVE (Per-Vertex Error). These metrics are particularly relevant to human mesh recovery, since many joints and vertices are involved in the process of recovering a 3D human mesh.

In addition to taking additional metrics for analysis, it may be useful to demonstrate how much more refined KITRO's output is in comparison to the unrefined data. This would involve rendering 3D human meshes of the original data by taking the unrefined shape, pose, and camera estimates, feeding them through SMPL, and taking the output vertices and faces to convert to a GIF rendering with the visualization script.

A particularly ambitious direction to take the project in would be to create versions of KITRO that are designed for the mesh recovery of things other than humans. For example, other animals, such as bears or cats, can have their skeleton's kinematic tree taken in place of the human skeleton's kinematic tree, and the decision tree used by KITRO would change to reflect that. For objects which lack bones, however, it is uncertain as to how cleanly the kinematic tree would fit for KITRO. The ability to flexibly refine meshes when given an appropriate kinematic tree would have interesting implications, since the subject broadens from humans to organisms with bones or joints. As an extension of this, it could also be useful to have a model that automatically generates a kinematic tree when provided with the 2D image of an organism's skeleton. This would mainly improve the accessibility of applying KITRO on subjects that aren't human.

Finally, it would be incredibly beneficial to have KITRO automatically detect the operating system and the presence of GPUs, as well as their corresponding drivers. The current model is limited to the CPU for users who are unable to get PyTorch to recognize their GPU. KITRO's current implementation relies solely on CUDA for GPU parallelization, which is incompatible with certain devices and operating systems, such as mac OS. Limiting KITRO to the CPU, especially when it's for creating the refined 3D mesh outputs, severely hinders its performance. For mac OS, PyTorch presents other, non-CUDA-based methods to improve performance, but significant revisions to KITRO would need to be made. Thus, it would be best to have KITRO automatically detect the drivers and operating system of the user, and determine which resources and methods it can use from there.