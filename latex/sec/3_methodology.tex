\section{Methodology}
\label{sec:methodolody}

Our project introduces several key changes: the restructuring of KITRO's output data to allow popular regression models to use it for training, the implementation of one such model that trains with KITRO's refinement parameters and predictions as additional features, and scripts which facilitate the generation of 3D mesh outputs as image files.

The restructured output of KITRO parameters and predictions is recorded and saved to a file. From before KITRO refinement is applied, parameters for the HMR model's theta, beta, and camera estimations, as well camera intrinsics, are all captured. After KITRO is applied, the newly refined theta, shape, camera, and vertex predictions, as well as 2D key points, are all captured. The pre-KITRO and post-KITRO values are combined and organized into one large collection of PyTorch tensors. From here, external models can convert the PyTorch tensors into Numpy arrays or whichever format is most suitable, and filter through specific feature columns as needed, or train using all of them.

Many considerations must be made before selecting a model to run on top of KITRO, since several well-known models are eliminated after accounting for certain constraints. Models which run with KITRO's refinement parameters and predictions as features must be regression models, since the predicted features are continuous, numerical values. Furthermore, models that scale poorly with dimensionality, i.e., large numbers of feature columns, cannot be considered, as the output of KITRO's predictions contains well over 20,000 feature columns. Lastly, as was mentioned in the paper where KITRO was first introduced, gradient-based models are noted to be less effective for refining 3D human meshes, due to inaccuracies which arise when attempting to correct proximal joints \cite{KITRO_2024}.

Our primary choice for the model to run on top of KITRO was the Random Forest regression model. Random Forest regression does not rely on Gradient Descent corrections that would otherwise interfere with creating 3D human meshes. Additionally, it does not suffer from working with high dimensionality, and it is simple to configure in relation to some other models, such as XGBoost, which requires extensive tuning. These aspects make the Random Forest regression model one of the more preferable choices to train with KITRO's output included.

In the project repository for KITRO, visualizing the 3D mesh outputs requires users to find additional dependencies to install and then program the scripts themselves. Introduced alongside the restructuring of KITRO's output and the regression model that runs on top of it, the mesh visualization scripts in this project serve to handle multiple sources of input. For visualizing KITRO as it's running, one script intercepts KITRO's outputs in batches and saves the outputs as PNG or GIF files. For visualizing the output of the combined Random Forest regression + KITRO model, another script takes a Numpy file with the predicted theta, beta, and camera values, and produces a GIF file of the new model's result. These two scripts facilitate the process of visually inspecting the generated 3D meshes for accuracy, and all the changes introduced by this project ultimately make it easier to understand the inner workings of human mesh recovery.