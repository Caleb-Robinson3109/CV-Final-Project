\section{Setup}
\label{sec:setup}

It seemed to our group that there were three sections of the project that could be worked on concurrently. The first section was the generation of the 3D mesh. Because there were prepocessed datasets, this person could build a 3D mesh visualizer on this data, and later run the same visualizer on the post-Random-Forest data.

The second person would be responsible for making the dataset contestable. Because the codebase was new to the team, understanding the given datasets and understanding how to transform it for use would be an important part of the project.

The third person would use scikit-learn to create a Random Forest model, train it, and test the model. Any additional scripts would be created as needed.

It was decided that the best way to distribute the files would be through GitHub. This is because the KITRO code was already hosted on GitHub, and it was most the common and streamlined choice for all the team members.

The last step to get the project started was for all of the team members to read and discuss the KITRO paper. This was crucial to make sure the scope and project were clear, and so that the team had an understanding on what was happening with KITRO under the hood. This would then transition into the teams understanding on how to build on top of the KITRO model.
